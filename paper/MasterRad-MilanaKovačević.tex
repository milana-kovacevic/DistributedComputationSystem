% Format teze zasnovan je na paketu memoir
% http://tug.ctan.org/macros/latex/contrib/memoir/memman.pdf ili
% http://texdoc.net/texmf-dist/doc/latex/memoir/memman.pdf
% 
% Prilikom zadavanja klase memoir, navedenim opcijama se podešava 
% veličina slova (12pt) i jednostrano štampanje (oneside).
% Ove parametre možete menjati samo ako pravite nezvanične verzije
% mastera za privatnu upotrebu (na primer, u b5 varijanti ima smisla 
% smanjiti 
\documentclass[12pt,oneside]{memoir} 

% Paket koji definiše sve specifičnosti master rada Matematičkog fakulteta
\usepackage[latinica]{matfmaster}

% Dodatni paketi

\usepackage{xcolor}

% Paket za prikazivanje C# koda
\usepackage{listings}

% paket za subfigure
\usepackage{caption}
\usepackage{subcaption}


\definecolor{bluekeywords}{rgb}{0,0,1}
\definecolor{greencomments}{rgb}{0,0.5,0}
\definecolor{redstrings}{rgb}{0.64,0.08,0.08}
\definecolor{xmlcomments}{rgb}{0.5,0.5,0.5}
\definecolor{types}{rgb}{0.17,0.57,0.68}

\lstset{language=[Sharp]C,
	captionpos=b,
	numbers=left,
	numberstyle=\tiny,
	frame=single,
	framesep=10pt,
	showspaces=false,
	showtabs=false,
	breaklines=true,
	showstringspaces=false,
	breakatwhitespace=true,
	tabsize=4,
	framexleftmargin=1.5em,
	xleftmargin=2em,
	escapeinside={(*@}{@*)},
	commentstyle=\color{greencomments},
	morekeywords={partial, var, value, get, set},
	keywordstyle=\color{bluekeywords},
	stringstyle=\color{redstrings},
	basicstyle=\ttfamily\footnotesize,
	classoffset=1, 
	morekeywords={ until, final, abstract, event, new, struct,
	as, explicit, null, switch,
	base, extern, object, this,
	bool, false, operator, throw,
	break, finally, out, true,
	byte, fixed, override, try,
	case, float, params, typeof,
	catch, for, private, uint,
	char, foreach, protected, ulong,
	checked, goto, public, unchecked,
	class, if, readonly, unsafe,
	const, implicit, ref, ushort,
	continue, in, return, using,
	decimal, int, sbyte, virtual,
	default, interface, sealed, volatile,
	delegate, internal, short, void,
	do, is, sizeof, while,
	double, lock, stackalloc,
	else, long, static,
	enum, namespace, string, begin,
	declare, function, algorithm, integer,
	procedure,
	>,<,.,;,,,-,!,=,~,
	await},
	classoffset=0,
}

\lstdefinestyle{yaml}{
     basicstyle=\color{blue}\footnotesize,
     rulecolor=\color{black},
     string=[s]{'}{'},
     stringstyle=\color{blue},
     comment=[l]{:},
     commentstyle=\color{black},
     morecomment=[l]{-}
 }


% Datoteka sa literaturom u BibTex tj. BibLaTeX/Biber formatu
\bib{references}

% Ime kandidata na srpskom jeziku (u odabranom pismu)
\autor{Milana Kovačević}
% Naslov teze na srpskom jeziku (u odabranom pismu)
\naslov{Razvoj platforme za distribuirano izračunavanje u oblaku}
% Godina u kojoj je teza predana komisiji
\godina{2022}
% Ime i afilijacija mentora (u odabranom pismu)
\mentor{prof. dr Saša \textsc{Malkov}\\ Univerzitet u Beogradu, Matematički fakultet}
% Ime i afilijacija prvog člana komisije (u odabranom pismu)
\komisijaA{prof. dr Filip \textsc{Marić}\\ Univerzitet u Beogradu, Matematički fakultet}
% Ime i afilijacija drugog člana komisije (u odabranom pismu)
\komisijaB{doc. dr Ivan \textsc{Čukić}\\ Univerzitet u Beogradu, Matematički fakultet}

% Datum odbrane (odkomentarisati narednu liniju i upisati datum odbrane ako je poznat)
% \datumodbrane{}

% Apstrakt na srpskom jeziku (u odabranom pismu)
\apstr{Rezime TODO}

% Ključne reči na srpskom jeziku (u odabranom pismu)
\kljucnereci{distribuirani sistemi, Azure tehnologije, oblak, Softver kao servis}

\begin{document}
% ==============================================================================
% Uvodni deo teze
\frontmatter
% ==============================================================================
% Naslovna strana
\naslovna
% Strana sa podacima o mentoru i članovima komisije
\komisija
% Strana sa posvetom (u odabranom pismu)
\posveta{Porodici i najbližima za strpljenje i podršku tokom studiranja}
% Strana sa podacima o disertaciji na srpskom jeziku
\apstrakt
% Sadržaj teze
\tableofcontents*

% ==============================================================================
% Glavni deo teze
\mainmatter
% ==============================================================================

% ------------------------------------------------------------------------------
\chapter{Uvod}
% ------------------------------------------------------------------------------
Uvod TODO

\section{Distriburani sistemi}

Distribuirani sistemi se sastoje od skupa fizički odvojenih mašina koje su međusobno povezane mrežom. Na ovim mašinama su pokrenute softverske jedinice koje međusobno dele odgovornost, poslove, komuniciraju i sinhronizuju se.

Prednosti ovako struktuiranog softvera su značajne. Neke od njih su:
\begin{enumerate}
\item Mogućnost postizanja visokog nivao paralelizacije prilikom raspodele poslova na mašine,
\item Deljenje resursa,
\item Prevazilaženje problema jednog mesta otkazivanja sistema (eng. \emph{single point of failure}). Ovim se povećava dostupnost softvera i njegova otpornost na greške,
\item Mogućnost skaliranja broja mašina uključenih u sistem, čime se sistem  može prilagoditi potrebama.
\item 
\item 
\end{enumerate}

Glavna mana distribuiranih sistema je se oni oslanjaju na transport poruka kroz mrežu. Prezasićenjem mreže može se doći do povećanog kašnjenja prilikom transporta informacije od jednog dela sistema do drugog. Dodatna mana predstavlja kako obezbediti sistem, što je teže uraditi nego na sistemima sa jednom mašinom. Na kraju, implementacija i održavanje distribuiranih sistema su načelno kompleksniji nego rad sa aotmičnim sistemima.

Distribuirani sistemi se mogu razvrstati u zavisnosti od načina povezivanja njegovih mašina. Postoje dva tipa arhitekture:
\begin{enumerate}
\item Klijent-server - TODO
\item Peer-to-peer - TODO
\end{enumerate}


%Тиме се добија софтвер који успешно враћа тражени резултат и над великим и захтевним улазима – што повећава његову употребљивост и трајност. 

\section{Distriburano izračunavanje}

Količina podataka u svetu kao i broj zahteva za njegovo obrađivanje neprekidno rastu. Obrada velikih podataka su postala potreba svakodnevnice. Međutim, često zbog fizičkih ograničenja nije moguće izvršiti izračunavanje u najednoj mašini, ili, ukoliko je to moguće, ono ne zadovoljava očekivane performanse. Sistemi za distribuirano izračunavanje predstavljaju rešenje za obradu velike količine podataka i potražnja za njima raste u koraku sa globalnom količinom podataka.

\section{Analiza postojećih sistema}
\label{sec:postojeca_resenja}

Analiza postojećih sistema TODO

Spark \cite{Spark}

Azure Functions pregled

neko rešenje u AWS?

\section{Opis korišćenih tehnologija i alata}

U ovom podpoglavlju biće opisane tehnologije i alati korišćenih prilikom programske realizacije sistema.

\subsection{Platforma Doker}
\label{subs:docker_platform}

Za pokretanje aplikacija je korišćena platforma Doker (eng. \emph{Docker}) \cite{Docker}. Prednosti korišćenja ove platforme su monogstuke. Za početak, ona razdvaja razvijanje aplikacije od infrastrukture na kojoj će biti pokrenuta. Aplikacija je spakovana u izolovano okruženje koje se naziva kontejner (eng. \emph{container}). Kontejneri sadrže sve (a u idealnom slučaju, samo) ono što je neophodno za pokretanje aplikacije, u vidu strukture koja se naziva slika (eng. \emph{image}). To čini da su kontejneri lagani za prenošenje, za razliku od virtuelnih mašina koje u teoriji mogu pružiti istu funkcionalnost. Spakovana aplikacija može biti pokrenuta neograničeni broj puta, u različitim okruženjima: prilikom ručnog i automatskog testiranja, u produkciji, itd. Prilikom pokretanja kontejnera, može se dodatno specificirati njegova konfiguracija (na primer, mapiranje portova).

Arhitektura platforme je predstavljena na slici \ref{fig:dockerarch}, preuzetoj iz zvanične dokumentacije \cite{DockerArch}. 

\begin{figure}[!ht]
  \centering
  \includegraphics[width=1.0\textwidth]{./images/docker_architecture.png}
  \caption{Arhitektura platforme Doker}
  \label{fig:dockerarch}
\end{figure}

Koraci potrebni da se napravi slika aplikacije se definišu u \emph{Dockerfile} datoteci. Svaka instrukcija u ovom fajlu kreira sloj slike (eng. \emph{layer}). U sličaju \emph{.NET Core} aplikacija, prvi korak je učitavanje željenog radnog okvira, zatim, (primera radi) pokretanje kompilacije izvornog koda i smeštanje izvršnog koda na željenu lokaciju. Krajnji korak je uglavnom definisanje komande za pokretanje aplikacije. Bitno je naglasiti da tokom razvijanja aplikacije, Doker prepoznaje koji slojevi slike koji su se promenili, te ponovo kreira samo njih i njihove naredne slojeve. Ovo čini generisanje slike efikasnom i brzom operacijom koja ne usporava programera, koji najčeće menja samo delove aplikacije. Kreirane slike se čuvaju u registru slika koji može biti lokalni ili negde u oblaku.

Kontejner sadrži specifikaciju operativnog sistema na kojem se pokreće slika, i on može biti \emph{Windows}, \emph{Mac} i \emph{Linux}.

Korišćeno okruženje za lokalno kreiranje i pokretanje kontejnera je \emph{Docker Desktop} prikazan na slici \ref{fig:dockerdesktop}.

\begin{figure}[!ht]
  \centering
  \includegraphics[width=1.0\textwidth]{./images/docker_desktop.png}
  \caption{Docker Desktop}
  \label{fig:dockerdesktop}
\end{figure}

% Kubernetes detalji
\subsection{Platforma Kubernetes}
\label{subs:kubernetes_platform}

Pokretanje sistema na praktičan i skalabilan način, omogućila je platforma \emph{Kubernetes} \cite{Kubernetes}. Ona pruža potrebnu infrastrukturu za pokretanje aplikacija zapakovanih u Doker kontejnere, kao i za upravljanje njima i ostalim pratećim delovima sistema.

Kubenetes omogućava:
\begin{enumerate}
\item Pronalaženje servisa koristeći DNS ime ili IP adresu,
\item Balansiranje saobraćaja kroz mrežu u zavisnosti od opterećenosti,
\item Korišćenje diska i drugih skladišta podataka,
\item Automatsko ažuriranje (eng. \emph{update}) verzija aplikacije kao i vraćanje na prethodnu verziju,
\item Upravljanje resursima i pakovanje kontejnera na mašine,
\item \emph{Self-healing} - automatski restart kontejnera koji ne ispunjavaju uslove zdravlja,
\item Upravljanje konfiguracijom, šiframa, sertifikatima i drugim osetljivim informacijama - uz samostalno ažuriranje bez potrebe za promenom slike kontejnera.
\end{enumerate}

Na slici \ref{fig:kubernetesarh} prikazana je arhitektura Kubernetes klastera, preuzeta iz zvanične dokumentacije \cite{KubernetesArchitecture}.

\begin{figure}[!ht]
  \centering
  \includegraphics[width=1.0\textwidth]{./images/kubernetes_architecture.png}
  \caption{Arhitektura klastera Kubernetes}
  \label{fig:kubernetesarh}
\end{figure}

Kubernetes klaster se sastoji iz dve celine: kontrolnog dela i skupa radnih mašina (eng. \emph{nodes}). Radne mašine na sebi pokreću \emph{mahune} (eng. \emph{pod}) u okviru kojih je smešten jedan ili više kontejner, u okviru kojih se nalazi aplikacija. Mahuna je najmanja jedinica koju je moguće kreirati i pokrenuti na klasteru, a kontejneri unutar njega imaju istu specifikaciju i dele lokalnu mrežu.

Kako bi pokrenute aplikacije bile dostupne na mreži, kreira se apstrakcija koja se naziva \emph{Servis}. Ova apstrakcija povezuje grupu mahuna i definiše način na koji im se pristupa: koristeći jedinstveno DNS ime ili IP adresu (eng. \emph{Service endpoints}). Slanjem poruka na ovu adresu, Kubernetes sam usmerava i balansira saobraćaj ka mahunama, od kojih svaka ima jedinstvenu IP adresu.

U okviru svake radne mašine je pokrenuto još nekoliko sistemskih procesa: 
\begin{enumerate}
\item \emph{kuberlet} - Zadužen za pokretanje kontejnera na mašini, kao i za praćenje rada kontejnera i njihovog zdravlja.
\item \emph{kube-proxy} - Zadužen za podešavanje pravila mreže koja omogućavaju slanje i prijem poruka mahunama, koristeći specifikacije definisane servisima.
\item \emph{Container runtime} - Zadužen za pokretanje kontejnera.
\end{enumerate}

Kontrolni deo klastera služi za upravljanje klasterom, a funkcionalnosti su obezbeđene kroz nekoliko komponenti:
\begin{enumerate}
\item \emph{kube-apiserver} - Server API koji prima zahteve upućene klasteru.
\item \emph{etcd} - Služi kao skladište podatka u klasteru.
\item \emph{kube-scheduler} - Zadužen za smeštanje novih mahuna na mašine u skladu sa dostupnim i traženim resursima, kao i drugim specifikacijama (na primer, međusobni afinitet servisa).
\item \emph{kube-controller-manager} - Sastoji se od nekoliko kontrolera:
	\begin{enumerate}
	\item \emph{node-controller} - Omogućava reagovanje u slučajevima kada mašine postanu nedostupne.
	\item \emph{job-controller} - Izvšava dodatne poslove na klasteru.
	\item \emph{endpoints-controller} - Podešava endpointe, tj. omogućava pronalaženje servisa i mahuna koje obuhvata. Ažurira adrese u slučaju promena (dodavanja i brisanja mahuna).
	\item \emph{Servce Account \& Token Controller} - Resursi u okviru klastera se mogu podeliti po imenskim prostorima (eng. \emph{namespace}). Ovi kontroleri omogućavaju podešavaju prava pristupa u okviru imenskih prostora.
	\end{enumerate}
\item \emph{cloud-controller-manager} - Predstavlja vezu između klastera i snabdevača resursa u oblaku (eng.\emph{cloud provider}).
\end{enumerate}
Sve aplikacije su pokrenute na fizičkim (ili virtuelnim) mašinama koje su date na raspolaganju klasteru tokom njegovog kreiranja ili ažuriranja. Neki kontejneri/mahune mogu biti pokrenuti na istoj mašini, ali to zavisi od potražnje i raspodele resursa, kao i od drugih specifikacija. Jedna od prednosti Kubernetesa je što on vodi računa o tome gde je koji proces pokrenut, te pokušava da obezbedi najbolju otpornost na greške u sistemu (na primer, restartovanje mašine) i time obezbedi visoku dostupnost pokrenutih servisa.

Dodatno, visoka dostupnost servisa se obezbeđuje kroz kontrolisani proces ažuriranja aplikacija, koristeći mehanizam koji se naziva eng. \emph{rolling update}. Ovaj sistem podrazumeva da se inkrementalno zamenjuju mahune, sa novokreiranim mahunama koje su pokrenute sa novom verzijom. Tokom ažuriranja, saobraćaj do mahuna se automatski raspoređuje samo ka dostupnim mahunama.

Za definisanje resursa i njihovih specifikacija, kao i konfiguracije na klasteru koriste se datoteke sa ekstenzijom \emph{yaml}. Prosleđivanjem ovih datoteka klasteru se rade promene na klasteru.

Za komunikaciju sa kontrolnim delom Kubernetes klastera se koristi kilijent \emph{kubectl}. Neke od komandi su predstavljene u nastavku.

\begin{verbatim}
// Opšti oblik komande
kubectl [command] [TYPE] [NAME] [flags]
// Kreiranje resursa i njihovih specifikacija definisane u deploy.yaml
kubectl apply -f deploy.yaml
// Izlistavanje servisa na klasteru
kubectl get svc
// Izlistavanje podova
kubectl get pods
// Pristup bash konzoli na mašini u okviru koje je pokrenut pod
kubectl exec mypod-m96mk -it mypod-m96mk  -- /bin/bash
// Podešava automatsko skaliranje broja mahuna
kubectl autoscale deployment computenode --cpu-percent=50 --min=1 --max=10
\end{verbatim}

\subsection{Platforma \emph{Microsoft Azure}}
\label{sub:azureplatform}

Programska implementacija sistema je uključila njegovo pokretanje u oblaku, a za to je korišćena platforma \emph{Microsoft Azure} \cite{Azure}. Ova platforma pruža veliki broj softverskih i infrastrukturnih rešenja, kao i propratnih servisa i mogućnosti koje poboljšavaju celokupno iskustvo korišćenja.

Najrelevantniji resursi korišćeni za pokretanje sistema u na \emph{Azure} platformi su:
\begin{enumerate}
\item \emph{Azure Kubernetes Service} (skr. AKS) \cite{AKS} - Resurs koji predstavlja \emph{Kubernetes} klaster.
\item \emph{Azure SQL Database} (skr. AzureSQLDB) \cite{AzureSQLDB} - Resurs koji predstavlja relacionu bazu podataka.
\item \emph{Azure Active Directory} \cite{AAD}  (skr. AAD)- Resurs koji omogućava kreiranje identiteta (npr. korisnike, grupe) i njihovo njegovo podešavanje (prava pristupa, login...).
\item \emph{Azure Monitor} (skr. AM) \cite{AzureMonitor} - Skup propratnih funkcionalnosti koje pružaju uvid u ponašanje Azure resursa. Sakuplja logove i metrike, daje alate za njihovo analiziranje, kao i mogućnost uzbunjivanja (eng. \emph{alerts}) i reagovanja na dešavanja od interesa. Ima podršku i za tehnike mašinskog učenja nad sakupljenim podacima. Više o integraciji sa \emph{Azure Monitor}-om se nalazu u poglavlju \ref{chp:pracenjemetrika}.
\end{enumerate}

\subsection{Dodatni alati}

\subsubsection{Swagger / OpenAPI}

\emph{OpenAPI} \cite{OpenAPI} je specifikacija \emph{REST API}-a (eng. \emph{Application Programming Interface}) koja ne zavisi od programskog jezika. \emph{Swagger} \cite{Swagger} predstavlja skup alata koji koriste \emph{OpenAPI} specifikaciju.

Korišćenjem skupa alata \emph{Swagger}, moguće je dokumentovati serverski web \emph{REST API} u standardizovanom formatu, u datoteci \emph{json}. Za pravljenje ove specifikacije, koristi se alat \emph{Swashbuckle} \cite{Swashbuckle}.

Za generisanje klijenta u rednom okviru \emph{.NET Core}, koristi se alat \emph{NSwag} \cite{NSwag}. Na osnovu prethodno generisane \emph{json} datoteke, on generiše klasu koja sadriži \emph{HTTP} klijenta i potrebnu dokumentaciju. Ovo je praktičan i efikasan način da se automatski generiše kod klijenta pomoću kojeg se šalju zahtevi serveru.



% ------------------------------------------------------------------------------
\chapter{Sistem DCS}
\label{chp:sistemdcs}
% ------------------------------------------------------------------------------

Kao sastavni deo master rada razvijen je projekat nazvan \emph{Distributed Computation System} (skr. \emph{DCS}).

Glavni zadatak sistema je da omogući korisniku obradu podataka na distribuirani način.
U jednostavnom slučaju, sistem obrađuje prosleđeni niz atomičnih poslova, a zatim agregira njihove rezultate. Ovo predstavlja distribuiranu implementaciju funkcionalnosti eng. \emph{map-reduce}.
U naprednijem slučaju, sistem DCS je moguće usko specijalizovati za obradu određenog tipa posla. Tada se prilikom implementacije koristi i poznavanje prirode posla, što omogućava da se on podeli na podjedinice. Krajnji rezultat se takodje dobija agregiranjem podrezultata, s tim da podjedinice posla mogu biti međusobno zavisne, te je za celokupno izvršavanje posla potrebno pratiti plan distribuiranog izvršavanja.

TODO primer nekog kompleksnijeg posla

Jedan od glavnih ciljeva sistema DCS je njegova modularnost i primenljivost. On je uz jednostavne izmene proširiv kako bi podržao različite tipove poslova koje su od značaja njegovim korisnicima, pritom, oslanjajući se na zajedničku infrastrukturu za obradu poslova. Ova infrastruktura je opisana u narednim poglavljima.

%, a predstavlja glavni deo sistema koji služi za zakazivanje i praćenje poslova, kao i čuvanje rezultata izvršavanja.

% ------------------------------------------------------------------------------
\section{Funkcionalnosti}
\label{chp:opisfunkc}
% ------------------------------------------------------------------------------

TODO 

BPMN dijagrami su prikazani na slikama \ref{fig:bpmn_korisnik} i \ref{fig:bpmn_otkazivanje}. TODO

\begin{figure}[!ht]
  \centering
  \includegraphics[width=1\textwidth]{./images/BPMN_dijagram_saradnje_korisnik_dsc.png}
  \caption{BMPN dijagram saradnje - Izvršavanje posla TODO ROTIRAJ}
  \label{fig:bpmn_korisnik}
\end{figure}

\begin{figure}[!ht]
  \centering
  \includegraphics[width=1\textwidth]{./images/BPMN_dijagram_saradnje_otkazivanje.png}
  \caption{BMPN dijagram saradnje - Otkazivanje posla}
  \label{fig:bpmn_otkazivanje}
\end{figure}



\begin{figure}[!ht]
  \centering
  \includegraphics[width=0.6\textwidth]{./images/dijagram_slucajeva_upotrebe_korisnik.png}
  \caption{UML dijagram slučajeva upotrebe - Korisnik}
  \label{fig:slucajupotrebe_korisnik}
\end{figure}


%Pre opisivanja slucajeva upotrebe bi trebalo da se oni ukratko navedu, sa nazivima i kratkim opisima
%	- lakse je razumeti pojedinacan slucaj ako se razume kontekst u kome se on nalazi
%	- mozda ima smisla staviti i neki dijagram aktivnosti ili bolje BPMN koji povezuje slucajeve upotrebe, tj. njihove ulaze i izlaze

U ovom poglavlju je prikazan pregled funkcionalnosti sistema za distribuirano izvršavanje DCS. 
Glavna funkcionalnost sistema je obrada korisnikovih zahteva za izvršavanje poslova. 
Funkcionalnosti su opisane u narednim odeljcima kroz slučajeve upotrebe, iz ugla korisnisnika i iz ugla dva tipa administratora: bezbednosnog i klaster administratora.

\subsection{Izvršavanje poslova na zahtev}
Slučajevi upotrebe iz ugla korisnika su prikazani na slici \ref{fig:slucajupotrebe_korisnik}, a detaljnije su opisani u narednim odeljcima.

\begin{figure}[!ht]
  \centering
  \includegraphics[width=0.6\textwidth]{./images/dijagram_slucajeva_upotrebe_korisnik.png}
  \caption{UML dijagram slučajeva upotrebe - Korisnik}
  \label{fig:slucajupotrebe_korisnik}
\end{figure}

% Kreiranje posla
\subsubsection{Dodavanje posla}
\begin{enumerate}
\item Naziv: Dodavanje posla
\item Akter: Korisnik koji želi da izvrši posao nad ulaznim podacima
\item Kratak opis: Korisnik šalje zahtev za kreiranje posla. Sistem validira zahtev i vraća potvrdu o uspešnosti primanja zahteva.
\item Preduslovi: Korisnik ima pristup internetu. Korisnik ima neophodna prava da bi poslao zahtev sistemu. Sistem je u funkciji.
\item Postuslovi: Sistem je evidentirao novi zahtev za posao i prosledio ga na asinhrono izvršavanje.
\item Tok događaja:
	\begin{enumerate}
	\item \label{zak:pravizahtev}Korisnik pravi zahtev za dodavanje posla. Zahtev se sastoji od:
		\begin{enumerate}
		\item Specifikacije tipa posla,
		\item Niza ulaznih podataka
		\end{enumerate}
	\item \label{zak:saljezahtev} Korisnik šalje zahtev sistemu preko interneta koristeći definisani interfejs.
	\item \label{zak:proveraprava} Sistem proverava korisnikova prava.
	\item \label{zak:validacijazahteva} Sistem validira novopridošli zahtev za izvršavanje posla.
	\item Sistem evidentira novi zahtev za izvršavanje u bazi.
	\item \label{zak:asinhzakazivanje} Sistem stavlja zahtev u red za asinhrono izvršavanje.
	\item Korisnik dobija potvrdu da je posao prihvaćen i identifikator po kojem je posao zaveden u sistemu.
	\end{enumerate}
\item Alternativni tok događaja:
	\begin{enumerate}
	\item Korisnik nema prava da podnese zahtev. Ukoliko u koraku \ref{zak:proveraprava} korisnik nema neophodna prava, sistem odbija zahtev uz odgovarajuću grešku. Proces se nastavlja u koraku \ref{zak:saljezahtev} glavnog toka.
	\item Zahtev je neispravan. Ukoliko u koraku \ref{zak:validacijazahteva} sistem prepozna neispravan zahtev, odbija ga uz odgovarajuću grešku. Neispravan zahtev može biti:
		\begin{enumerate}
		\item Neispravna specifikacija tipa posla. Zahtevani tip posla mora biti podržan od strane sistema,
		\item Format ulaznih podataka nije odgovarajući.
		\end{enumerate}
Proces se nastavlja u koraku \ref{zak:pravizahtev} glavnog toka.
	\item Sistem je preopterećen i nema dovoljno resursa da zakaže novi posao. Ukoliko u koraku \ref{zak:asinhzakazivanje} sistem proceni da nema dovoljno resursa za izvršavanje posla, on odbija zahtev uz grešku da je sistem preopterećen i da korisnik pokuša kasnije. Proces se nastavlja u koraku \ref{zak:saljezahtev} glavnog toka.
	\end{enumerate}
\item Podtokovi: /
\item Specijalni zahtevi: /
\item Dodatne informacije: /
\end{enumerate}

% Dohvatanje rezultata
\subsubsection{Preuzimanje rezultata posla}
\begin{enumerate}
\item Naziv: Preuzimanje rezultata posla
\item Akter: Korisnik koji želi da pruzme rezultate prethodno zakazanog posla
\item Kratak opis: Korisnik šalje zahtev za preuzimanje rezultata izvršavanja. Sistem validira zahtev i vraća tražene rezultate.
\item Preduslovi: Korisnik ima pristup internetu. Korisnik ima neophodna prava da bi poslao zahtev sistemu. Sistem je u funkciji. Posao je izvršen i rezultati su dostupni u bazi.
\item Postuslovi: Sistem je prosledio rezultate izvršavanja korisniku.
\item Tok događaja:
	\begin{enumerate}
	\item \label{rez:pravizahtev} Korisnik pravi zahtev za preuzimanje rezultata posla. Zahtev se sastoji od identifikatora posla za koji želi da preuzme rezultate.
	\item \label{rez:saljezahtev} Korisnik šalje zahtev sistemu preko interneta koristeći definisani interfejs.
	\item \label{rez:proveraprava} Sistem proverava korisnikova prava.
	\item \label{rez:validacijazahteva} Sistem validira novopridošli zahtev za preuzimanje rezultata posla.
	\item \label{rez:sinhslanjerez} Sistem šalje rezultate korisniku.
	\item Korisnik dobija rezultate.
	\end{enumerate}
\item Alternativni tok događaja:
	\begin{enumerate}
	\item Korisnik nema prava da podnese zahtev. Ukoliko u koraku \ref{rez:proveraprava} korisnik nema neophodna prava, sistem odbija zahtev uz odgovarajuću grešku. Proces se nastavlja u koraku \ref{rez:saljezahtev} glavnog toka.
	\item Zahtev je neispravan. Ukoliko u koraku \ref{rez:validacijazahteva} sistem prepozna da posao sa datim identifikatorom ne postoji, odbija zahtev uz odgovarajuću grešku. Proces se nastavlja u koraku \ref{rez:pravizahtev} glavnog toka.
	\end{enumerate}
\item Podtokovi: /
\item Specijalni zahtevi: Korisnik ima neophodno znanje za generisanje zahteva i korišćenje interfejsa sistema.
\item Dodatne informacije: Ukoliko je posao uspešno izvršen, rezultat izvršavanja se sastoji od traženog rezultata izračunavanja. Ukoliko je posao neuspešno obrađen, rezultat se sastoji od informacije o grešci.
\end{enumerate}

% Otkazivanje
\subsubsection{Otkazivanje posla}
\begin{enumerate}
\item Naziv: Otkazivanje posla
\item Akter: Korisnik koji želi da otkaže prethodno zakazani posao
\item Kratak opis: Korisnik šalje zahtev za otkazivanje posla. Sistem validira zahtev i vraća potvrdu o uspešnosti otkazivanja.
\item Preduslovi: Korisnik ima pristup internetu. Korisnik ima neophodna prava da bi poslao zahtev sistemu. Sistem je u funkciji.
\item Postuslovi: Sistem je otkazao posao i ažurirao evidenciju posla u bazi.
\item Tok događaja:
	\begin{enumerate}
	\item \label{otk:pravizahtev} Korisnik pravi zahtev za otkazivanje posla. Zahtev se sastoji od identifikatora posla koji želi da otkaže.
	\item \label{otk:saljezahtev} Korisnik šalje zahtev sistemu preko interneta koristeći definisani interfejs.
	\item \label{otk:proveraprava} Sistem proverava korisnikova prava.
	\item \label{otk:validacijazahteva} Sistem validira novopridošli zahtev za otkazivanje posla.
	\item \label{otk:sinhotkazivanje} Sistem sinhrono otkazuje sve operacije povezane sa poslom.
	\item Korisnik dobija potvrdu da je posao otkazan.
	\end{enumerate}
\item Alternativni tok događaja:
	\begin{enumerate}
	\item Korisnik nema prava da podnese zahtev. Ukoliko u koraku \ref{otk:proveraprava} korisnik nema neophodna prava, sistem odbija zahtev uz odgovarajuću grešku. Proces se nastavlja u koraku \ref{otk:saljezahtev} glavnog toka.
	\item Zahtev je neispravan. Ukoliko u koraku \ref{otk:validacijazahteva} sistem prepozna da posao sa datim identifikatorom ne postoji ili nije aktivan, odbija zahtev uz odgovarajuću grešku. Proces se nastavlja u koraku \ref{otk:pravizahtev} glavnog toka.
	\end{enumerate}
\item Podtokovi: /
\item Specijalni zahtevi: /
\item Dodatne informacije: /
\end{enumerate}

% Pregled poslova
\subsubsection{Pregled poslova}
\begin{enumerate}
\item Naziv: Pregled poslova
\item Akter: Korisnik koji želi da pruzme listu evidentiranih poslova u sistemu.
\item Kratak opis: Korisnik šalje zahtev za izlistavanje evidentiranih poslova. Sistem validira zahtev i vraća tražene informacije o evidentiranim poslovima.
\item Preduslovi: Korisnik ima pristup internetu. Korisnik ima neophodna prava da bi poslao zahtev sistemu. Sistem je u funkciji. Posao je izvršen i rezultati su dostupni u bazi.
\item Postuslovi: Sistem je prosledio rezultate izvršavanja korisniku.
\item Tok događaja:
	\begin{enumerate}
	\item \label{pregledp:konstruisezahtev} Korisnik pravi zahtev za izlistavanje poslova.
	\item \label{pregledp:saljezahtev} Korisnik šalje zahtev sistemu preko interneta koristeći definisani interfejs.
	\item \label{pregledp:proveraprava} Sistem proverava korisnikova prava.
	\item \label{pregledp:slanjeliste} Sistem šalje listu evidentiranih poslova dostupnih u bazi.
	\item Korisnik dobija listu evidentiranih poslova.
	\end{enumerate}
\item Alternativni tok događaja:
	\begin{enumerate}
	\item Korisnik nema prava da podnese zahtev. Ukoliko u koraku \ref{pregledp:proveraprava} korisnik nema neophodna prava, sistem odbija zahtev uz odgovarajuću grešku. Proces se nastavlja u koraku \ref{pregledp:saljezahtev} glavnog toka.
	\end{enumerate}
\item Podtokovi: /
\item Specijalni zahtevi: /
\item Dodatne informacije: Rezultujuća lista poslova sadrži informacije o svakom evidentiranom poslu, i to:
	\begin{enumerate}
	\item Identifikator posla
	\item Status posla
	\item Vreme početka zvršavanja posla
	\item Vreme završetka izvršavanja posla
	\end{enumerate}
\end{enumerate}

% Administratorovi slučajevi upotrebe
\subsection{Podešavanje sistema}
Dodatne funkcionalnosti sistema uključuju slučajeve upotrebe u kojima je učesnik administrator. Ovo uključuje podešavanje bezbednosti i prava pristupa sistemu, kao i prilagođavanje sistema kako bi mogao na što efikasniji način da obradi korsnikove zahteve.

% Pristup podacima je ograničen i neophodno je da mogu da im pristupe samo autorizovane osobe.  -- prebaci negde

% AAD Admin
\subsubsection{Podešavanja prava pristupa}
Prava pristupa sistemu, tj. delovima sistema dodeljuje administrator za bezbednost. Prava pristupa pojedinačnim delovima sistema se kontrolišu kroz različta prava koja mogu biti dodeljena korisniku. Tipovi prava su:
	\begin{enumerate}
	\item Iskustvo krajnjeg korisnika - Mogućnost slanja zahteva sistemu kroz definisani javni interfejs,
	\item Pravo za praćenje rada sistema - Pristup telemetriji i logovima za praćenje rada sistema,
	\item Administrator klastera - Pristup klasteru za devops akcije,
	\item Administrator bezbednosti.
	\end{enumerate}
Slučajevi upotrebe iz ugla administratora za bezbednost su prikazani na slici \ref{fig:slucajupotrebe_aadadmin}. Ovih slučajevi upotrebe su jednostavni i intuitivni, te nisu opisani u nastavku.

\begin{figure}[!ht]
  \centering
  \includegraphics[width=0.6\textwidth]{./images/dijagram_slucajeva_upotrebe_administrator_sistema_aadadmin.png}
  \caption{UML dijagram slučajeva upotrebe - Administrator za bezbednost}
  \label{fig:slucajupotrebe_aadadmin}
\end{figure}


% Podešavanje AKS klastera
\subsubsection{Podešavanje klastera}
Administrator klastera ima mogućnost da menja konfiguraciju sistema kako bi ga prilagodio potrebama krajnjeg korisnika. To uključuje horizontalno i vertikalno skaliranje sistema. Horizontalno skaliranje sistema podrazumeva menjanje broja pokrenutih servisa koji se koriste tokom izvršavanja posla. Vertikalno skaliranje podrazumeva ažuriranje konfiguracije kojom se dodeljuju resursi servisima (dostupna memorija i procesorsko vreme).

Slučajevi upotrebe iz ugla administratora klastera su prikazani na slici \ref{fig:slucajupotrebe_admin_klastera}. Detalji ovih slučajeva upotrebe su konceptualno jasni, a uključuju poznavanje implementacionih detalja, te nisu dalje razrađivani.

\begin{figure}[!ht]
  \centering
  \includegraphics[width=0.6\textwidth]{./images/dijagram_slucajeva_upotrebe_administrator_klastera.png}
  \caption{UML dijagram slučajeva upotrebe - Administrator klastera}
  \label{fig:slucajupotrebe_admin_klastera}
\end{figure}

% ------------------------------------------------------------------------------
\section{Arhitektura sistema}
% ------------------------------------------------------------------------------

Sistem DCS je implementiran po uzoru na arhitekturu klijent-server.

Centralni deo sistema DCS se sastoji od dva tipa aplikacija: Kontrolne jedinice (eng. \emph{Control Node}, skr. CtrlN) i jedinice za izračunavanje (eng. \emph{Compute Node}, skr. CmpN). 
U sistemu postoji tačno jedna aplikacija kontrolne jedinice, koja je zadužena za dve logički odvojene celine: primanje zahteva od korisnika i orkestriranje izvršavanja prethodno zakazanih poslova.
Sa druge strane, u zavisnosti od potreba, DCS se sastoji od jedne ili više jedinica za izračunavanje koje su zadužene za izvršavanje atomičnih (nedeljivih) poslova. CmpN prima zahteve za izvršavanje atomičnih poslova od CtrlN.

Arhitektura sistema je prikazana na slici \ref{fig:arhitektura}.

\begin{figure}[!ht]
  \centering
  \includegraphics[width=1.0\textwidth]{./images/arhitektura_sistema_dijagram_komponenti.png}
  \caption{UML dijagram komponenti - Arhitektura sistema}
  \label{fig:arhitektura}
\end{figure}

Arhutekturalni obrazac korišćen prilikom razvoja sistema je u Prezentacija-Apstrakcija-Kontrola (eng. \emph{Presentation-Abstraction-Control}, skr. PAC). Sistem je hijerarhijski podeljen na agente CtrlN i CmpN. CtrlN je agent višeg novoa, koji je pozvezan sa CmpN agentima. Ovakvom arhitekturom se uvodi podela odgovornosti različitih delova servisa.

PAC je slojevita arhitektura, te svaki od agenata CtrlN i CmpN ima PAC slojeve. Prvi sloj aplikacija je \emph{prezenter} koji predstavlja REST API. Ovaj sloj aplikacije je okrenut ka spolja i definiše interfejs za komunikaciju sa servisom. U slučaju CtrlN je to API preko kog prima korisnikove zahteve, a u slučaju CmpN je to API preko kog prima on zahteve od CtrlN. Drugi sloj aplikacija je \emph{kontroler} koji predstavlja centralni deo agenta gde je glavna logika. Treći sloj aplikacija je \emph{apstrakcija} i predstavlja model podataka i interfejs prema bazi podataka. 


% ------------------------------------------------------------------------------
\section{Implementacija}
% ------------------------------------------------------------------------------

Sistem DCS je javno dostupan na servisu GitHub na adresi \href{https://github.com/milana-kovacevic/DistributedComputationSystem}{https://github.com/milana-kovacevic/DistributedComputationSystem}\label{githubdsc}. Za implementaciju je korišćen programski jezik C\#, i radni okvir \emph{.NET Core 6.0}. Korišćeno je razvojno okruženje \emph{Microsoft Visual Studio Community 2022} i operativni sistem \emph{Windows 10}.

Glavna logika distribuiranog izvršavanja nalazi se u okviru kontrolne jedinice. Jedinice za izvršavanje su jednostavni servisi koji imaju jasan cilj koji podrazumeva izvršavanje atomičnog posla i vraćanje rezultata izvršavanja. U nastavku, razrađuje se implementacija CtrlN komponente (ukoliko nije naglašeno drugačije).

Kontrolna jedinica se sastoji od dve komponente koje se nazivaju: Frontend (skr. FE) i Distribuirani orkestrator (skr. DO). Detaljnije, zaduženja CtrlN su, redom:
\begin{enumerate}
\item Frontend
	\begin{enumerate}
	\item Autentifikacija klijenta,
	\item Primanje i validacija zahteva,
	\end{enumerate}
\item Distribuirani orkestrator
	\begin{enumerate}
	\item \label{podela_planiranje}Podela zahteva na potposlove i pravljenje plana izvršavanja,
	\item \label{orkestriranje}Orkestiranje izvršavanja potposlova i agregacija podrezultata,
	\item \label{cuvanje_u_bazi}Čuvanje rezultata u bazi ili nekom drugom skladištu.
	\end{enumerate}
\end{enumerate}

Moguća stanja kroz koja prolazi posao su prikazana na slici \ref{fig:stanjaposla} a u nastavku je objašnjen način njegove obrade.

\begin{figure}[!ht]
  \centering
  \includegraphics[width=1.0\textwidth]{./images/dijagram_stanja_posao.png}
  \caption{UML dijagram stanja - Posao}
  \label{fig:stanjaposla}
\end{figure}

Kada je FE primio i evidentirao zahtev za izvršavanje posla, on ubacuje posao u red za izvršavanje. U slučaju da je sistem zauzet obradom drugih poslova, novi zahtev će čekati dok sistem ne postane spreman da ga obradi. Za implementaciju reda za čekanje, korišćeno je postojeće rešenje bezbedno pri radu sa nitima \emph{BlockingCollection} \cite{BlockingCollection}. Posebna nit aplikacije uzima poslove iz reda i šalje ih jednog po jednog na dalju obradu komponenti DO.

Navedena zaduženja \ref{podela_planiranje} i \ref{orkestriranje} uključuju glavni deo logike distribuiranog izvršavanja. Pre nego što se počne sa izvršavanjem, neophodno je podeliti posao na potposlove i definisati njihove zavisnosti. Rezultat ovoga je predstavljen u vidu stabla u kojem listovi predstavljaju početne atomične poslove, unutrašnji čvorovi agregacije podrezultata, a grane prenos međurezultata. Koren predstavlja krajnje rešenje. U slučaju da korisnikov zahtev već sadrži niz atomičnih poslova, podela početnog zahteva je trivijalna. Na osnovu stabla zavisnosti, pravi se plan izvršavanja. Sličan način planiranja koriste baze podataka \cite{SQLServerInternals}, kao i sistemi poput Sparka \cite{Spark} predstavljen u odeljku \ref{sec:postojeca_resenja}. Plan izvršavanja zavisi od tipa posla, a u slučaju \emph{map-reduce} funkcionalnosti podrazumeva se da se svaki atomični posao nezavisno obrađuje (funkcijom \emph{map}), a krajnje rešenje se dobija definisanom agregacijom atomičnih rezultata (funkcijom \emph{reduce}). Na osnovu plana izvršavanja, komponenta DO sprovodi dalje izvršavanje posla: šalje atomične poslove preko mreže do izabranih CmpN na izvršavanje. Zakazivanje i izvršavanje atomičnih poslova se dešava asinhrono i ono ne blokira DO da uzme naredni posao iz reda i započne njegovu obradu. Dobijeni rezultati se šalju odvojenoj komponenti koja zna da agregara rezultate na osnovu plana izvršavanja. U zavisnosti od kompleksnosti agregacije, i ona može da se prosledi CmpN na izračunavanje.

DO dodatno prati uspešnost izvršavanja poslova i u slučaju uspešnog završetka, upisuje rezultat u bazu. Ukoliko dođe do greške prilikom izvršavanja, posao može da se ponovi ili da se evidentira greška u krajnjem rezultatu. Ponovno pokretanje poslova je potrebno kako bi se povećao stepen pouzdanosti izvršavanja. U distribuiranim sistemima treba imati u vidu da često može doći do problema u komunikaciji među servisima, bilo zbog preopterećenosti mreže ili eventualnih restarta individualnih aplikacija. Ovo može uzrokovati prolaznu grešku, koju je moguće prevazići novim pokušajem. Ukoliko i nakon ponovnog izvršavanja dođe do greške, posao se završava i kao krajnji rezultat se upisuje greška.

Klijent ima mogućnost da otkaže aktivan posao. Ova operacija uključuje i otkazivanje svih njegovih potposlova koji su u procesu izvršavanja ili koji čekaju na izvršavanje, te se ukupno otkazivanje dešava globalno u celom sistemu.
Kada zahtev za otkazivanje stigne do komponente DO, on proverava status posla i status njegovih potposlova. Ukoliko u tom trenutku postoji barem jedan atomični posao koji čeka na izvršavanje ili se izvršava, DO prekida dalje zakazivanje i asinhrono otkazuje poslove koji su u toku izvršavanja. Tada se ažurira status posla i on čeka da se otkažu svi atomični poslovi. Kada sistem globalno otkaže sva izvršavanja, posao je otkazan i klijentu se vraća potvrda o upešnosti.

TODO dijagram klasa DO

\subsection{Baza podataka}

Sistem DSC koristi relacionu bazu podataka za evidentiranje poslova. DCS ažurira stanja u kojem se nalazi posao kako bi korisnik mogao da prati šta se dešava u sistemu. Dodatno, baza podataka služi i za čuvanje rezultata i međurezultata poslova.

ER (eng. \emph{Entity-Relation}) dijagram modela posla je prikazan na slici \ref{fig:erposao}.

\begin{figure}[!ht]
  \centering
  \includegraphics[width=1.0\textwidth]{./images/uml_er_dijagram_posao.png}
  \caption{ER dijagram entiteta u bazi podataka}
  \label{fig:erposao}
\end{figure}


\subsection{Autentifikacija}

Za autentifikaciju klijenta je korišćen servis AAD \cite{AAD} opisan u uvodnom poglavlju u odeljku \ref{sub:azureplatform}. Korisniku su dodeljena prava korišćenja servisa time što je njegov identitet član grupe korisnika definisane u okviru servisa AAD.

Proces autentikacije je prikazan na slici \ref{fig:autentifikacija}. Korisnik pre kontaktiranja sistema DSC šalje zahtev servisu AAD kako bi preuzeo token za autentifikaciju. Ovaj token ima rok trajanja (uobičajeno je 60 minuta) i specifičan je za autentifikaciju na DSC sistem. Kada klijent ima token, prosleđuje ga u \emph{header} delu HTTP zahteva u polju za autentifikaciju. Sistem DCS prima zahtev i pre nego što počne da ga obrađuje, izvrši validaciju AAD tokena. Token sadrži otisak (eng. \emph{thumbprint}) koji garantuje njegovu validnost. Ukoliko je token ispravan, nastavlja se sa obradom zahteva, a ukoliko nije, korisnikov zahtev se odbija uz odgovarajuću grešku.

Za bezbednu komunikaciju u produkcionom okruženju, forsiran je protokol HTTPS.

\begin{figure}[!ht]
  \centering
  \includegraphics[width=1.0\textwidth]{./images/autentikacija_uml_dijagram_sekvence.png}
  \caption{UML dijagram sekvence - Autentifikacija}
  \label{fig:autentifikacija}
\end{figure}


\section{Pokretanje u oblaku}

Opisana implementacija dobija smisao i punu moć pokretanjem sistema DCS na klasteru Kubernetes predstavljenom u uvodu u delu \ref{subs:kubernetes_platform}. Komponente Frontend i Compute Node se pokreću u okviru kontejnera Doker (uvodni deo, odeljak \ref{subs:docker_platform}). Izbor radnog okvira \emph{.NET Core} omogućava pokretanje servisa na različitim platformama \emph{Windows}, \emph{Mac} i \emph{Linux}, bez izmena izvornog koda. Za izradu sistema su korišćeni kontejneri Doker sa operativnim sistemom \emph{Linux}.

Za pokretanje sistema DCS u oblaku je korišćena platforma \emph{Microsoft Azure} \cite{Azure}. Za upravljanje i pristup resursima u okviru ove platforme koristi se portal \emph{Azure} \cite{AzurePortal}.

TODO još detalja o setapu

yaml primer

objasnjenje load balansinga ka CN

prikaz mahuna iz portala ili kubectl klijenta

DNS / https / ingress setup


Na adresi projekta na servisu GitHub \ref{githubdsc} se nalaze datoteke potrebne za kreiranje i konfigurisanje servisa pokrenutih u okviru klastera Kubernetes, kao i pomoćne skripte koje automatizuju proces postavljanja nove verzije aplikacija na klaster.


% ------------------------------------------------------------------------------
\section{Okvir za testiranje}
\label{chp:testiranjesistema}
% ------------------------------------------------------------------------------

Za izradu testova korišćen je radni okvir \emph{XUnit} \cite{XUnit}.

Svi testovi se nalaze na GitHub adresi projekta u poddirektorijumu \href{https://github.com/milana-kovacevic/DistributedComputationSystem/tree/main/tests}{tests}.

Radni okvir \emph{XUnit} ima podršku za pisanje dva tipa testova: činjenice (eng. \emph{Fact}) i teorije (eng. \emph{Theory}). Činjenice su testovi koji ne primaju argumente, dok teorije predstavljaju paremetrizovane testove. Specifikacijom atributa iznad tela funkcije teorijskog testa, definišu se ulazi nad kojima se test pokreće, što značajno olakšava generisanje testova sa različitim ulazima.

Struktura foldera sa testovima je prikazana na slici \ref{fig:testovi}, a na slici \ref{fig:testexplorer} se nalazi vizuelni prikaz rezultata izvršavanja testova u okruženju \emph{Microsoft Visual Studio}.
% (TODO nova slika sa jos testova)

\begin{figure}[!ht]
  \centering
  \includegraphics[width=0.8\textwidth]{./images/testovi.png}
  \caption{Struktura foldera sa testovima}
  \label{fig:testovi}
\end{figure}

\begin{figure}[!ht]
  \centering
  \includegraphics[width=0.8\textwidth]{./images/testexplorer.png}
  \caption{Pokretanje testova u okruženju \emph{Microsoft Visual Studio}}
  \label{fig:testexplorer}
\end{figure}


%subfigures primer
%\begin{figure}[ht]
%\begin{subfigure}{.5\textwidth}
%  \centering
%  \includegraphics[width=1\textwidth]{./images/testexplorer.png}
%  \caption{Struktura foldera sa testovima}
%  \label{fig:testovi2}
%\end{subfigure}
%\begin{subfigure}{.5\textwidth}
%  \centering
%  \includegraphics[width=1\textwidth]{./images/testovi.png}
%  \caption{Pokretanje testova iz \emph{Test Explorer}-a u okruženju \emph{Microsoft Visual Studio}}
%  \label{fig:testexplorer2}
%\end{subfigure}
%\caption{Put your caption here}
%\label{fig:fig}
%\end{figure}


Sistem je testiran na više nivoa, počevši od testiranja jedinica koda za svaku komponentu, zatim testova integracije, i, na najvišem nivou, funkcionalnih testova koji ukljucuju klaster testove.

\subsection{Testiranje jedinica koda}
Komponente \emph{Control Node} i \emph{ComputeNode} prate odgovarajući testovi jedinica koda. Ovi testovi se nalaze u zasebnom projektu. Tokom izrade testova jedinica koda, korišćen je i radni okvir \emph{Moq} \cite{Moq} za "podmetanje" vrednosti u zavisnim delova koda. U nastavku se nalazi slika \ref{fig:testscheduler} sa primerom testa jedinice koda u okviru komponente CtrlN.

\begin{figure}[h!]
\centering
\begin{lstlisting}
public class DistributedSchedulerTests
{
	private readonly ServiceProvider serviceProvider;
	private readonly Mock<IComputeNodeClientWrapper> mockedComputeNodeClient = new();

	public DistributedSchedulerTests()
	{
		// Configure services using common bootstraper for tests
		var services = new ServiceCollection();
		TestBootstraper.ConfigureServices_Frontend(services);

		// Setup mocked ComputeNodeClient
		mockedComputeNodeClient.Setup(m => m.RunAsync(It.IsAny<int>(), It.IsAny<int>(), It.IsAny<FrontendAtomicJobType>(), It.IsAny<string>()))
			.Returns(Task.FromResult(new FrontendAtomicJobResult()));
		services.AddScoped((services) => mockedComputeNodeClient.Object);

		services.AddScoped<IScheduler, DistributedScheduler>();
		serviceProvider = services.BuildServiceProvider();
	}

	[Fact]
	public async Task ScheduleJobAsync_Success()
	{
		var scheduler = serviceProvider.GetService<IScheduler>();
		Assert.NotNull(scheduler);

		var jobToBeScheduled = UnitTestUtils.GetDummyJob();
		await scheduler.ScheduleJobAsync(jobToBeScheduled);

		mockedComputeNodeClient.Verify(client => client.RunAsync(It.IsAny<int>(), jobToBeScheduled.JobId, It.IsAny<FrontendAtomicJobType>(), It.IsAny<string>()), Times.AtLeastOnce());
		Assert.NotEmpty(scheduler.GetInProgressTasks());
	}
}
\end{lstlisting}
\caption{Primer testa koji koristi radni okvir \emph{Moq}}
\label{fig:testscheduler}
\end{figure}

\subsection{Integracioni testovi}

Postoje dva tipa integracionih testova:
\begin{enumerate}
\item Testovi integracije unutar komponenti,
\item Testovi integracije celih komponenti.
\end{enumerate} 

Testovi integracije unutar komponenti testiraju međusobne zavisnosti podkomponenti CtrlN i CmpN. Na primer, zavisnost FE i DO u okviru CtrlN.

Testovi integracije celih komponenti CtrlN i CmpN se pokreću nad lokalnim okruženjem, i očekuju da je okruženje pripremljeno za njihovo izvšavanje. U okviru kontejnera Doker potrebno je pokrenuti jednu CtrlN aplikaciju, kao i jednu CmpN aplikaciju, sa ranije definisanim adresama i portovima. Koristi se automatski generisana klijentska klasa \emph{DistributedCalculationSystemClient} nastala korišćenjem alata \emph{Swagger} u toku komipacije projekta sa testovima. Ovaj klijent šalje zahteve servisu CtrlN, a u telu testova se proverava da li je odgovor od sistema očekivan.

\subsection{Testovi nad klasterom}

Testovi nad klasterom testiraju sistem iz ugla korisnika u produkcionom okruženju. Oni se pokreću nad klasterom u oblaku, i u njima se takođe koristi generisani klijent. Dele se na nekoliko grupa:
\begin{enumerate}
\item Funkcionalni testovi - Testiraju osnovne funkcionalnosti servisa u produkcionom okruženju.
\item Stres-testovi - Skup testova koji se izvršavaju kako bi se analiziralo i unapredilo ponašanje sistema pod opterećenjem. 
\item Testovi performansi - Skup testova koji se pokreću nad servisom pokrenutim u oblaku, sa fokusom na praćenje vremena potrebnog da sistem odradi očekivani posao. Cilj ovog testiranja je uvid u performanse sistema u odnosu na resurse dodeljene klasteru.
\end{enumerate}

Na slici \ref{fig:teste2e} se nalazi primer koji testira funkcionalnost sistema od početka do kraja: zakazivanje posla, preuzimanje statusa, i provera rezultata.

\begin{figure}[h!]
\centering
\begin{lstlisting}
[Fact]
public async void RunJob_Success()
{
	var inputData = new Collection<AtomicJobRequestData>()
	{
		new AtomicJobRequestData() { InputData ="42" },
		new AtomicJobRequestData() { InputData ="142" },
	};
	string expectedTotalSum = "13";

	var request = new JobRequestData()
	{
		JobType = JobType.CalculateSumOfDigits,
		InputData = inputData
	};

	// Create job.
	var job = await _client.CreateAsync(request);
	Assert.NotNull(job);

	// Verify job is created.
	var jobFromSystem = await _client.JobsAsync(job.JobId);
	Assert.NotNull(jobFromSystem);

	// Poll and verify job state until it's successfully completed.
	await TestUtils.PollUntilSatisfied(
		job.JobId,
		(jobId) =>
		{
			var jobFromSys = _client.JobsAsync(jobId).GetAwaiter().GetResult();
			return jobFromSys.State == JobState.Succeeded;
		},
		timeout: defaultTimeout);

	// Verify aggregated result
	var jobResult = await _client.JobResultsAsync(job.JobId);
	Assert.Equal(string.Empty, jobResult.Error);
	Assert.Equal(expectedTotalSum, jobResult.Result);
	Assert.Equal(JobState.Succeeded, jobResult.State);

	// Delete job
	await _client.DeleteAsync(job.JobId);

	// Now getting job should throw 404.
	var exception = Assert.ThrowsAsync<ApiException>(async () => await _client.JobsAsync(job.JobId));
	Assert.Equal<int>((int)HttpStatusCode.NotFound, exception.Result.StatusCode);
}
\end{lstlisting}
\caption{Primer funkcionalnog testa}
\label{fig:teste2e}
\end{figure}

Stres-testovi i testovi performansi ispituju granice sistema. Oni su neophodni kako bi se na vreme razumela i unapredila uska grla sistema, kao i da bi se razumele mogućnosti sistema u skladu sa dodeljenim resursima.

\subsubsection{Testovi performansi nad klasterom sa različitim konfiguracijama}
TODO


% ------------------------------------------------------------------------------
\section{Praćenje rada sistema}
\label{chp:pracenjemetrika}
% ------------------------------------------------------------------------------

U ovom odeljku su prikazane mogućnosti praćenja rada sistema uz pomoć servisa \emph{Azure Monitor} za praćenje ponašanja resursa u oblaku. Kako su glavne komponente sistema pokrenute u okviru klastera \emph{Azure Kubernetes Service}, dat je pregled načina na koji se analiziraju podaci o radu klastera. Na analogan način se može pratiti i ponašanje baze podataka \emph{Azure SQL Database} koju koristi sistem.

\subsection{Metrike}

Metrike sistema predstavljaju koncizne informacije o njegovom radu u (skoro) realnom vremenu. Kroz metrike možemo da vidimo trenutnu iskorišćenost resursa klastera, status mašina, mahuna, kontejnera itd.

Na slici \ref{fig:aksinsights} je prikazan pogled na sekciju \emph{Insight} pripadajuceg klastera kojem se pristupa preko portala \cite{AzurePortal}.

\begin{figure}[!ht]
  \centering
  \frame{\includegraphics[width=1\textwidth]{./images/aks_insights.png}}
  \caption{Pregled metrika klastera}
  \label{fig:aksinsights}
\end{figure}

Na slici \ref{fig:clusterunhealthy} je prikazan pogled na \emph{Insight} sekciju pripadajućeg klastera u slučaju zdravih servisa 
% Na slici se vidi da je servis markiran uzvičnikom, jer nije dostignut željeni broj zdravih replika (0\% replika je zdravo).

\begin{figure}[!ht]
  \centering
  \frame{\includegraphics[width=1\textwidth]{./images/metrics_cluster.png}}
  \caption{Pogled na metrike o statusu klastera}
  \label{fig:clusterunhealthy}
\end{figure}

\subsection{Logovi}

Logovi predstavljaju informacije o radu sistema koje ispisuju pokrenuti servisi u tekstualnom formatu. Oni su automatski sakupljeni sa mašina i prosleđeni na mesto za čuvanje, gde im je moguće pristupiti kroz propratne alate.

Za pregled logova korišćen je alat \emph{Azure Log Analytics} dostupan u okviru servisa \emph{Azure Monitor}. Ovaj alat pruža moćan deskriptivni jezik kojim se mogu pretraživati, filtrirati, sortirati i analizirati logovi. Rezultate upita je takođe moguće vizuelizovati u vidu grafikona.

Na slici \ref{fig:computenodelogs} se nalazi prikaz stranice za pregled logova kontejnera koji sadrži Frontend aplikaciju.

\begin{figure}[!ht]
  \centering
  \frame{\includegraphics[width=1\textwidth]{./images/computenode_logs.png}}
  \caption{Pregled logova komponente ComputeNode}
  \label{fig:computenodelogs}
\end{figure}


\subsection{Uzbunjivači}

Korisnik koji ima pravo pristupa logovima i metrikama može da napravi automatsku uzbunu (eng. \emph{Alert}) u nekim situacijama od značaja. Ovaj servis periodično proverava dostupne metrike i logove, i proverava da li su se ispunjeni uslovi definisani tokom njegovog kreiranja. Ukoliko zaključi da postoji problem, podiže uzbunu na ranije definisan način (mejlom, SMS-om, pozivom, obaveštenjem) i, ukoliko postoje, automatski pokreće definisane akcije. Problemi u produkcionom okruženju se javljaju, i u tim slučajevima je cilj što pre reagovati i obezbediti da sistem bude zdrav i na raspolaganju korisnicima.

U nastavku se na slici \ref{fig:alertex} nalazi primer aktivnog upozoranja u slučaju grešaka koje bacaju komponente sistema.

\begin{figure}[!ht]
  \centering
  \frame{\includegraphics[width=1\textwidth]{./images/alert_unhandled_exceptions.png}}
  \caption{Aktivni uzbunjivač u slučaju neočkivanih grešaka prisutnim u logovima kontejnera}
  \label{fig:alertex}
\end{figure}

Na slici \ref{fig:alertmemory} nalazi primer aktivnog opšteg upozorenja u slučaju nedostatka memorije na mašinama. Analogno, postoji i uzbunjivač koji prati procenat iskorišćenosti dostupnog procesora.

\begin{figure}[!ht]
  \centering
  \frame{\includegraphics[width=1\textwidth]{./images/alert_memory_pressure.png}}
  \caption{Uzbunjivač u slučaju nedovoljno memorije u klasteru}
  \label{fig:alertmemory}
\end{figure}


% ------------------------------------------------------------------------------
\chapter{Diskusija i zaključak}
\label{chp:diskusijaizakljucak}
% ------------------------------------------------------------------------------

TODO par opstih recenica Pravci daljeg razvoja sistema

U inicijalnoj verziji servisa, samo CtrlN ima direktnu komunikaciju sa bazom.
Otkazivanje zakazanih poslova nije podržano u inicijalnoj verziji.

Prva verzija sistema DSC implementira \emph{map-reduce} funkcionalnost u kojom računa ukupan zbir cifara brojeva niza, gde svaki broj predstavlja jedan ulazni podatak za atomični posao. U naprednoj verziji sistema, plan bi bio predstavljan u vidu stabla u kojem listovi predstavljaju početne atomične poslove, unutrašnji čvorovi agregacije, a grane prenos međurezultata kroz mrežu. Ovakve planove koriste sistemi poput Sparka \cite{Spark} predstavljeni u uvodu \ref{sec:postojeca_resenja}.

Tip posla može i da preusmeri pisanje rezultata na proizvoljnu lokaciju.
Ukoliko su ulazni podaci ili rezultat veliki, te ih nije praktično čuvati u bazi podataka, DCS je moguće proštiriti kako bi koristio neko drugo skladište za pristup ulaznim podacima i za čuvanje rezultata.

\section{Dalji razvoj sistema}
U ovom poglavlju je predloženo nekoliko ideja u kojem smeru bi predstavljeni sistem za izračunavanje mogao da se razvija.

\subsection{Generalizacija tipa poslova}

Prikazani sistem u inicijalnoj implementaciji ume da obrađuje samo poslove koji se sastoji od niza atomičnih poslova. Dodavanje podrške za obradu novih tipova posla je moguća uz sitnije izmene, čija je suština u dodavanju klase koja izvršava specifični atomični posao. Ova klasa implementira \href{https://github.com/milana-kovacevic/DistributedComputationSystem/blob/main/src/ComputeNode/Executors/ISpecificJobExecutor.cs}{\emph{ISpecificJobExecutor} interfejs} u projektu jedinice za izvršavanje. Izbor koji tip posla će sledeći biti implementiran zavisi od potražnje na tržištu.

Potpunu generalizaciju obrade atomičnih jedinica posla moguće je izvesti proširivanjem korisničkog interfejsa za zakazivanje posla, tj. proširenjem interfejsa da mu ulazni parametar bude izvorni kod funkcije koju je potrebno izvršiti nad atomičnim poslovima. Tada bi korisnik, pored ulaznih podataka, prosledio i deo koda koji želi da se izvrši nad atomičnim poslovima na sistemu. Za podršku ovog slučaja upotrebe, potrebno je da sistem zna da prevede izvorni kod funkcije i rezultujuću funkciju prosledi jedinici za izvršavanje. Ovaj pristup podrazumeva bezbednosne provere unetog koda, kako bi se osigurali od napada na klaster.

Dalja unapređenja bi mogla da uključe obradu poslova koji nisu nužno specificirani nizom svojih atomičnih podjedinica. Ovo podrazumeva da programer razume suštinu izvršavanja novog tipa posla, te generiše plan za distribuirano izvršavanje. Plan može da uključuje i agregaciju podrezultata zbog eventualnih zavisnosti potposlova.


\subsection{Podrška za različite frontende}

Implementirani frontend daje korisniku na korišćenje REST API. Dalji razvoj servisa bi mogao da podrži i druge klijent - server načine komunikacije, uključujući komunikaciju preko \emph{Web Socketa}. Takođe, moguće je izraditi \emph{Web} platformu i vizuelni korisnički interfejs za lakše korišćenje sistema.

Sistem je moguće proširiti da uključuje podršku i za druge tipove autentifikacije. Osnovna autentikacija je uključila integrisanje sa provajderom identiteta \emph{Azure Active Directory}. Ovo je moguće unaprediti dodavanjem odvojenu komponente za podršku za druge načine autentikacije (na primer, koriščenjem naloga \emph{Google}).

Navedena poboljšanja bi dosta opteretila današnju implementaciju frontenda. Sa daljim razvojem u ovom delu servisa, trebalo bi razdvojiti implementaciju Frontend servisa od komponente za distribuirano zakazivanje, i nastaviti razvoj imajući u vidu pristup razvijanja kroz mikro-servise. To bi uključilo izradu drugih Frontend servisa koji bi komunicirali sa svojim klijentima, a poslove slali na centralni servis zadužen za distribuirano orkestriranje.

\subsection{Napredno zakazivanje poslova}

Urađena implementacija se, prilikom izvršavanja atomičnih jedinica posla na jedinicama za izvršavanje, oslanja na Kubernetes podršku za balansiranje zahteva koji idu kroz mrežu do servisa. Napredniji način zakazivanja poslova bi uključio prepoznavanje adresa jedinica za izvšavanje (tj. njihovih podova) a zatim tu informaciju iskoristio za pametnije usmeravanje poslova na izvršavanje na tim adresama. U kalkulacije bi uključio i procene kompleksnosti posla, kao i dostupne resurse.

Ova implementacija je dosta zahtevnija ali pruža potencijalnu priliku za bolje performanse. Međutim, pre početka same implementacije, potrebno je izvšiti opsežno testiranje i analiziranje postojućeg sistema kako bi se opravdala potreba za novim okrestratorom distribuiranog izvršavanja. Ukoliko osnovna varijanta algoritma za orkestriranje ne zadovoljava kriterijume performansi, onda ima smisla raditi ovakva unapređenja.

Ukoliko je unapređenje opravdano, za početak, potrebno je napraviti infrastrukturu za pronalaženje adresa u okviru komponente koja vrši zakazivanje poslova. Dodatno, treba implementirati algoritam koji procenjuje kompleksnost posla u vidu potrebnih resursa za njegovo izračunavanje. Na kraju, potrebno je i implementirati novi orkestrator. Kada je ovo sve urađeno, potrebno je uraditi testiranje sistema sa novom implementacijom, kao i AB testiranje sistema kako bi se doneli zaključci koji je princip bolji i zašto.


\section{Zaključak}
Zaključak, TODO

% ------------------------------------------------------------------------------
% Literatura
% ------------------------------------------------------------------------------
\literatura

% ==============================================================================
% Završni deo teze i prilozi
\backmatter
% ==============================================================================

% ------------------------------------------------------------------------------
% Biografija kandidata
\begin{biografija}
  \textbf{Milana Kovačević} je rođena u Zrenjaninu, 29. novembra 1995. godine. Osnovno i srednje obrazovanje (Zrenjaninska gimnazija, prirodno-matematički smer) završila je u rodnom gradu, uz sticanje diplome "Vuk Karadžić". Takođe je završila nižu muzičku školu, instrument klavir.
2014. godine je upisala osnovne studije na modulu Informatika na Matematičkom fakultetu Univerziteta u Beogradu. Osnovne studije je završila 2017. godine sa prosečnom ocenom 9.86, kao primalac stipendije Dositeja. Master akademske studije upisala je 2017. godine takođe na Matematičkom fakultetu na modulu Informatika. Položila je sve ispite predviđene planom i programom master akademskih studija sa prosečnom ocenom 9.15.

Nakon završetka osnovnih studija, nastavlja da se paralelno razvija i u industriji, radom u kompaniji Microsoft. Tokom rada se susreće sa sistemima za obradu i čuvanje podataka u okviru Azure plaforme, a radom na jednom od njih stiče i praktično znanje o distribuiranim sistemima i tehnologijama za rad u oblaku.

\end{biografija}
% ------------------------------------------------------------------------------

\end{document}
